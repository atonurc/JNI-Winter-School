\usepackage{amsmath,amssymb,amsthm}
\usepackage{latexsym}
\usepackage{mathrsfs}
\usepackage[usenames,svgnames,dvipsnames]{xcolor}
\usepackage{hyperref}
\usepackage[nameinlink]{cleveref}
\usepackage{textcomp}
% \usepackage{enumerate}
\usepackage[shortlabels]{enumitem}
\usepackage{dsfont}
\usepackage{makecell}
\usepackage[textsize=scriptsize,shadow]{todonotes}
\usepackage{mathtools}
\usepackage{microtype}
\usepackage{epigraph}
\setlength\epigraphrule{0pt}
\setlength\epigraphwidth{8cm}
% \setcounter{chapter}{-1}
\usepackage[normalem]{ulem}
\usepackage{stmaryrd}
\usepackage{wasysym}
% \usepackage{mdsymbol}
% \usepackage{esint}
% \usepackage[charter]{mathdesign}
% \usepackage{pdfMsym}
% \input pdfmsym
% \pdfmsymsetscalefactor{11}
\usepackage{multirow}
\usepackage{prerex}
\usepackage{csquotes}
\usepackage{graphicx,tikz}
\usepackage{caption}
\usepackage{parallel}
\usepackage{tikz-cd}
\usepackage{asymptote}
\usepackage{pgfplots, pst-node}
\usepackage{pst-plot}
\usetikzlibrary{positioning}
\usetikzlibrary{shapes}
\usetikzlibrary{plotmarks}
\usetikzlibrary{decorations.markings}
\usetikzlibrary{arrows.meta,bending}
\usetikzlibrary{patterns}
% \usepackage{genyoungtabtikz}
% \usepackage{ytableau} LAGBE NA 
\usepgfplotslibrary{colormaps}
\pgfplotsset{
compat=newest,
colormap={mycolormap}{color=(darkgray) color=(lightgray) color=(gray)}
}
% \graphicspath{{./images/}}

\definecolor{azuree}{HTML}{007FFF}
\definecolor{Cobalt}{HTML}{0047AB}




\newcommand{\cbrt}[1]{\sqrt[3]{#1}}
\newcommand{\vect}[1]{\left\langle #1 \right\rangle}
\newcommand{\floor}[1]{\left\lfloor #1 \right\rfloor}
\newcommand{\ceil}[1]{\left\lceil #1 \right\rceil}
\newcommand{\mailto}[1]{\href{mailto:#1}{\texttt{#1}}}
\newcommand{\ol}{\overline}
\newcommand{\ul}{\underline}
\newcommand{\wt}{\widetilde}
\newcommand{\wh}{\widehat}
\newcommand{\eps}{\varepsilon}
\newcommand{\empt}{\varnothing}
\newcommand{\les}{\leqslant}
\newcommand{\smth}{C^{\infty}}
\newcommand{\smtj}{C^{\infty}}
\newcommand{\catname}{\mathsf}
\newcommand{\cat}{\mathcal}
\newcommand{\cone}{\mathbf{Cone}}
\newcommand{\cocone}{\mathbf{Cocone}}
\newcommand{\limc}{\varprojlim}
\newcommand{\clim}{\varinjlim}
\newcommand{\hrulebar}{
\par\hspace{\fill}\rule{0.95\linewidth}{.7pt}\hspace{\fill}
\par\nointerlineskip \vspace{\baselineskip}
}
\newcommand{\half}{\frac{1}{2}}



\DeclareMathOperator{\cis}{cis}
\DeclareMathOperator{\Int}{Int}
\DeclareMathOperator{\Bd}{Bd}
\DeclareMathOperator{\bd}{bd}
\DeclareMathOperator{\Lk}{Lk}
\DeclareMathOperator{\St}{St}
\DeclareMathOperator{\Sd}{Sd}
\DeclareMathOperator{\sd}{sd}
\DeclareMathOperator{\Ext}{ext}
\DeclareMathOperator{\id}{id}
\DeclareMathOperator{\cl}{cl}
\DeclareMathOperator{\Cl}{Cl}
\DeclareMathOperator{\supp}{supp}
\DeclareMathOperator{\Supp}{Supp}
\DeclareMathOperator{\Diff}{Diff}
\DeclareMathOperator{\Grp}{Group}
\DeclareMathOperator{\idd}{\mathds{1}}
\DeclareMathOperator{\ran}{range}
\DeclareMathOperator{\diam}{diam}
\DeclareMathOperator{\Arg}{Arg}
\DeclareMathOperator{\diag}{diag}
\DeclareMathOperator{\ev}{eval}
\DeclareMathOperator{\fun}{Fun}
\DeclareMathOperator{\vol}{vol}
\DeclareMathOperator{\grad}{grad}
\DeclareMathOperator{\curl}{curl}
\DeclareMathOperator{\divg}{div} % because \div doesnt work
\DeclareMathOperator*{\lcm}{lcm}




\newcommand{\CC}{\mathbb C}
\newcommand{\FF}{\mathbb F}
\newcommand{\NN}{\mathbb N}
\newcommand{\QQ}{\mathbb Q}
\newcommand{\RR}{\mathbb R}
\newcommand{\EE}{\mathbb E}
\newcommand{\HH}{\mathbb H}
\newcommand{\LL}{\mathbb L}
\newcommand{\ZZ}{\mathbb Z}
\newcommand{\PP}{\mathbb P}
\newcommand{\II}{\mathbb I}
\newcommand{\KK}{\mathbb K}
\newcommand{\dd}{\mathrm d}
\renewcommand{\d}{\mathrm d}

\newcommand{\A}{\mathcal A}
\newcommand{\s}{\mathcal S}

\newcommand{\w}{\wedge}
\renewcommand{\t}{\otimes}
\newcommand{\vf}{\mathfrak X}
\newcommand{\sy}{\mathfrak S} 


\newcommand{\RP}{\mathbb R P}

\newcommand{\inv}{^{-1}}


\newcommand{\abs}[1]{\left\lvert #1 \right\rvert}
\newcommand{\norm}[1]{\left\lVert #1 \right\rVert}
\newcommand{\dang}{\measuredangle} %% Directed angle
\newcommand{\ray}[1]{\overrightarrow{#1}}
\newcommand{\seg}[1]{\overline{#1}}
\newcommand{\arc}[1]{\wideparen{#1}}
\newcommand{\mcal}[1]{\mathcal{#1}}
\newcommand{\mt}[1]{\mathtt{#1}}
\newcommand{\ve}[1]{\mathbf{#1}}
\newcommand{\fin}{\text{fin}}
\newcommand{\tl}[1]{\overset{\sim}{#1}}


\renewcommand{\qedsymbol}{$\blacksquare$}


\newenvironment{subproof}[1][Proof]{%
\begin{proof}[#1] \renewcommand{\qedsymbol}{$\square$}}%
{\end{proof}}


\newenvironment{soln}{\begin{proof}[Solution]}{\end{proof}}

\newcommand{\liff}{\leftrightarrow}
\newcommand{\lthen}{\rightarrow}
\newcommand{\opname}{\operatorname}
\newcommand{\surjto}{\twoheadrightarrow}
\newcommand{\injto}{\hookrightarrow}

\renewcommand{\Re}{\opname{Re}}
\renewcommand{\Im}{\opname{Im}}

\DeclareMathOperator{\im}{im} % Image
\DeclareMathOperator{\Img}{Im} % Image
\DeclareMathOperator{\coker}{coker} % Cokernel
\DeclareMathOperator{\Coker}{Coker} % Cokernel
\DeclareMathOperator{\Ker}{Ker} % Kernel
\DeclareMathOperator{\rank}{rank}
\DeclareMathOperator{\rk}{rk}
\DeclareMathOperator{\tr}{Tr}
\DeclareMathOperator{\Tr}{Tr}
\DeclareMathOperator{\nul}{nullity}
\DeclareMathOperator{\dist}{dist}
\DeclareMathOperator{\Hom}{Hom}
\DeclareMathOperator{\Aut}{Aut}
\DeclareMathOperator{\End}{End}
\DeclareMathOperator{\Ob}{Ob}
\DeclareMathOperator{\sub}{Sub}
\DeclareMathOperator{\dom}{dom}
\DeclareMathOperator{\cod}{cod}
\DeclareMathOperator{\op}{op}
\DeclareMathOperator{\rep}{Rep}
\DeclareMathOperator{\res}{Res}
\DeclareMathOperator{\ind}{Ind}
\DeclareMathOperator{\sgn}{sgn}
\DeclareMathOperator{\sym}{Sym} 
\DeclareMathOperator{\spann}{span} % because \span doesnt work
\DeclareMathOperator{\Cok}{Coker} % Cokernel
\DeclareMathOperator{\cok}{cok} % Cokernel
\newcommand{\del}{\partial}

% matrix linear groups
\DeclareMathOperator{\GL}{GL}
\DeclareMathOperator{\SL}{SL}
\DeclareMathOperator{\SO}{SO}
\DeclareMathOperator{\SU}{SU}
\DeclareMathOperator{\OO}{O}

\hypersetup{
colorlinks,
linkcolor={blue},
citecolor={blue!50!black},
urlcolor={blue!80!black}
}



% THESE ARE FOR HEADERS AND FOOTERS 


\usepackage[headsepline]{scrlayer-scrpage}
\renewcommand{\headfont}{}
\addtolength{\textheight}{3.14cm}
\setlength{\footskip}{0.5in}
\setlength{\headsep}{10pt}
%
\automark[chapter]{chapter}

%
% \rohead{\footnotesize\thepage}
% \lehead{\footnotesize\thepage}
\lohead{\footnotesize \leftmark}
\chead{}
\rofoot{}
\refoot{}
\lefoot{}
\lofoot{}


% 	THESE LINES INDICATE HOW THE CHAPTERS AND SECTIONS AND SUBSECTIONS LOOK LIKE
\renewcommand*{\sectionformat}{\color{blue}\S\thesection\autodot\enskip}
\renewcommand*{\subsectionformat}{\color{blue}\S\thesubsection\autodot\enskip}
\renewcommand{\thesubsection}{\thesection.\roman{subsection}}

\addtokomafont{chapterprefix}{\raggedleft}
\RedeclareSectionCommand[beforeskip=0.5em]{chapter}
\renewcommand*{\chapterformat}{%
\mbox{\scalebox{1.5}{\chapappifchapterprefix{\nobreakspace}}%
\scalebox{2.718}{\color{blue}\thechapter\autodot}\enskip}}

% \addtokomafont{partprefix}{\rmfamily}
% \renewcommand*{\partformat}{\color{blue}\scalebox{2.5}{\thepart}}


\usepackage{thmtools}
\usepackage[framemethod=TikZ]{mdframed}

\theoremstyle{definition}
\mdfdefinestyle{mdbluebox}{%
roundcorner = 10pt,
linewidth=1pt,
skipabove=12pt,
innerbottommargin=9pt,
skipbelow=2pt,
nobreak=true,
linecolor=blue,
backgroundcolor=TealBlue!5,
}
\declaretheoremstyle[
headfont=\sffamily\bfseries\color{MidnightBlue},
mdframed={style=mdbluebox},
headpunct={\\[3pt]},
postheadspace={0pt}
]{thmbluebox}

\mdfdefinestyle{mdredbox}{%
linewidth=0.5pt,
skipabove=7pt,
frametitleaboveskip=5pt,
frametitlebelowskip=0pt,
skipbelow=2pt,
frametitlefont=\bfseries,
innertopmargin=4pt,
innerbottommargin=8pt,
% nobreak=true,
linecolor=RawSienna,
backgroundcolor=Salmon!5,
}
\declaretheoremstyle[
headfont=\bfseries\color{RawSienna},
mdframed={style=mdredbox},
headpunct={\\[3pt]},
postheadspace={0pt},
bodyfont=\color{black},
]{thmredbox}

\mdfdefinestyle{mdneelbox}{%
skipabove=8pt,
linewidth=2pt,
rightline=false,
leftline=true,
topline=false,
bottomline=false,
linecolor=RoyalBlue,
backgroundcolor=RoyalBlue!5,
}
\declaretheoremstyle[
headfont=\bfseries\sffamily\color{RoyalBlue!70!black},
bodyfont=\normalfont,
spaceabove=2pt,
spacebelow=1pt,
mdframed={style=mdneelbox},
headpunct={\\[3pt]},
]{thmneelbox}
%%%%%%%%
\mdfdefinestyle{mdazurebox}{%
skipabove=8pt,
linewidth=2pt,
rightline=false,
leftline=true,
topline=false,
bottomline=false,
linecolor=azuree,
backgroundcolor=azuree!5,
}
\declaretheoremstyle[
headfont=\bfseries\sffamily\color{azuree!70!black},
bodyfont=\normalfont,
spaceabove=2pt,
spacebelow=1pt,
mdframed={style=mdazurebox},
headpunct={\\[3pt]},
]{thmazurebox}
%%%%%%
\mdfdefinestyle{mdcobaltbox}{%
skipabove=8pt,
linewidth=2pt,
rightline=false,
leftline=true,
topline=false,
bottomline=false,
linecolor=Cobalt,
backgroundcolor=Cobalt!5,
}
\declaretheoremstyle[
headfont=\bfseries\sffamily\color{Cobalt!70!black},
bodyfont=\normalfont,
spaceabove=2pt,
spacebelow=1pt,
mdframed={style=mdcobaltbox},
headpunct={\\[3pt]},
]{thmcobaltbox}

\newtheorem{remark}{Remark}[chapter]
\newtheorem{example}{Example}[chapter]


\declaretheorem[%
style=thmcobaltbox,name=Theorem,numberwithin=chapter]{theorem}
\declaretheorem[style=thmazurebox,name=Lemma,sibling=theorem]{lemma}
\declaretheorem[style=thmneelbox,name=Proposition,sibling=theorem]{proposition}
\declaretheorem[style=thmbluebox,name=Corollary,sibling=theorem]{corollary}
% \declaretheorem[style=thmredbox,name=Example,numberwithin=chapter, sibling=ex]{example}
% I dont do this anymore.

\mdfdefinestyle{mdgreenbox}{%
skipabove=8pt,
linewidth=2pt,
rightline=false,
leftline=true,
topline=false,
bottomline=false,
linecolor=ForestGreen,
backgroundcolor=ForestGreen!5,
}
\declaretheoremstyle[
headfont=\bfseries\sffamily\color{ForestGreen!70!black},
bodyfont=\normalfont,
spaceabove=2pt,
spacebelow=1pt,
mdframed={style=mdgreenbox},
headpunct={: },
]{thmgreenbox}

\mdfdefinestyle{mdkalabox}{%
skipabove=8pt,
linewidth=2pt,
rightline=false,
leftline=true,
topline=false,
bottomline=false,
linecolor=black,
backgroundcolor=RedViolet!5!gray!5,
}
\declaretheoremstyle[
headfont=\bfseries\sffamily\color{black},
bodyfont=\normalfont,
spaceabove=1pt,
spacebelow=1pt,
mdframed={style=mdkalabox},
headpunct={.},
]{thmkalabox}
\declaretheorem[style=thmkalabox,name=Definition,numberwithin=chapter]{defn}

\mdfdefinestyle{mdblackbox}{%
skipabove=8pt,
linewidth=3pt,
rightline=false,
leftline=true,
topline=false,
bottomline=false,
linecolor=black,
backgroundcolor=RedViolet!5!gray!5,
}
\declaretheoremstyle[
headfont=\bfseries,
bodyfont=\normalfont\small,
spaceabove=0pt,
spacebelow=0pt,
mdframed={style=mdblackbox}
]{thmblackbox}

\theoremstyle{theorem}
\declaretheorem[name=Claim,style=thmgreenbox, numbered = yes]{claim}


\mdfdefinestyle{mdbalackbox}{%
roundcorner = 10pt,
linewidth=1pt,
skipabove=8pt,
innerbottommargin=9pt,
skipbelow=0pt,
linecolor=black,
nobreak=true,
%backgroundcolor=TealBlue!5,
}
\declaretheoremstyle[
headfont=\sffamily\bfseries\color{Black},
mdframed={style=mdbalackbox},
headpunct={: }  ,
postheadspace={0pt}
]{thmbalackbox}
\declaretheorem[style=thmbalackbox,name=Case,numbered=no]{case*}
\declaretheorem[style=thmbalackbox,name=Case]{case}
\newtheorem{subcase}{Subcase}[case]



\mdfdefinestyle{mdremarkbox}{%
roundcorner = 10pt,
linewidth=1pt,
skipabove=8pt,
innerbottommargin=9pt,
skipbelow=0pt,
linecolor=purple,
nobreak=true,
backgroundcolor=purple!2,
}
\declaretheoremstyle[
headfont=\sffamily\bfseries\color{purple},
mdframed={style=mdremarkbox},
headpunct={. }  ,
postheadspace={0pt}
]{thmremarkbox}

\declaretheorem[style=thmremarkbox,name=Remark,numberwithin=chapter, sibling=remark]{remark*}
\theoremstyle{definition}
\newtheorem*{fact}{Fact}
\newtheorem*{abuse}{Abuse of Notation}
\newtheorem*{notation}{Notation}
\newtheorem*{ques}{Question}
\newtheorem{exercise}{Exercise}[chapter]
\newtheorem{problem}{Problem}[chapter]
\newtheorem{axiom}{Axiom}






\usepackage[backend=biber,style=numeric]{biblatex}
\addbibresource{references.bib}




\usepackage{mathrsfs}

\newcommand\MM{\mathscr M}
\newcommand{\UU}{\mathscr U}
\newcommand{\Bb}{\mathscr B}
\newcommand{\Ss}{\mathscr S}
\newcommand{\Cs}{\mathscr C}
\newcommand{\Dd}{\mathscr D}
\newcommand{\VV}{\mathscr V}
\newcommand{\Nn}{\mathscr N}
\newcommand{\Aa}{\mathscr A}
\renewcommand{\emptyset}{\varnothing}
